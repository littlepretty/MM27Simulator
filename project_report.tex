\documentclass[12pt]{article}  % [12pt] option for the benefit of aging markers

% amssymb package contains more mathematical symbols
\usepackage{amssymb,amsthm}

% graphicx package enables you to paste in graphics
\usepackage{graphicx}
\usepackage{auto-pst-pdf}

% embed source inside latex
\usepackage[procnames]{listings}
\usepackage{color}

%%%%%%%%%%%%%%%%%%%%%%%%%%%%%%%%%
%
%    Page size commands.  Don't worry about these
%
\setlength{\textheight}{220mm}
\setlength{\topmargin}{-10mm}
\setlength{\textwidth}{150mm}
\setlength{\oddsidemargin}{0mm}

%%%%%%%%%%%%%%%%%%%%%%%%%%%%%%%%%%%%%%%%%%%%%%%%%%%%%%%%%%%%%%%
%
%    Definitions of environments for theorems etc.
%
\newtheorem{theorem}{Theorem}[section]          % Theorems numbered within sections - eg Theorem 2.1 in Section 2.
\newtheorem{corollary}[theorem]{Corollary}      % Corollaries etc. will be counted as Theorems for numbering
\newtheorem{lemma}[theorem]{Lemma}              % eg Lemma 3.1, ... Theorem 3.2, ... Corollary 3.3.
\newtheorem{proposition}[theorem]{Proposition}
\newtheorem{conjecture}[theorem]{Conjecture}

\theoremstyle{definition}
\newtheorem{definition}[theorem]{Definition}

\theoremstyle{remark}
\newtheorem{remark}[theorem]{Remark}
\newtheorem{example}[theorem]{Example} 

%%%%%%%%%%%%%%%%%%%%%%%%%%%%%%%%%%%%%%%%%%%%%%%
%
%        Preamble material specific to your essay
%
\title{Simulation of M/M/2/2+5 Queueing System}
\author{Jiaqi Yan, Ping Liu\\
CS555 Project\\
supervised by
Edward Chlebus}

\begin{document}
\maketitle

% \newpage                     % optional page break
\begin{abstract}
In this project, we simulate and analyze the M/M/2/2+5 queueing system. 
To generate generate packet arriving time and processing time, we use Python's built-in \textit{random} module, which is validated carefully.
Then we apply the \textbf{Welch graphical procedure} to eliminate the warm-up period in the simulation.
With the stationary region, we then analyze the system's properties such as blocking probability and mean number of packet in the system.
The 90\% confidence interval for these properties are also given.
\end{abstract}

% optional page break
\newpage
\tableofcontents

% optional page break
\newpage
\section{Discrete Event Simulation}\label{sec:des}

%\begin{center}
%\begin{tabular}{|r|r|r|r|r|r|r|}        % 7 columns, each right-justified
%\hline                                  % horizontal line between rows
 %& $R_1$ & $R_2$ & $R_3$ & $R_4$ & $R_5$ & $R_6$ \\ % header row
%\hline
%$R_1$ &   & 3 &   &   &    & \\
%\hline
%$R_2$ & 4 &   & 6 &   &    & \\
%\hline
%$R_3$ &   &   &   & 7 &    & \\
%\hline
%$R_4$ &   &   &   &   & 11 & \\
%\hline
%$R_5$ &   &   &   &   &    & 9 \\
%\hline
%$R_6$ &   & 8 & 5 &   &    & \\
%\hline
%\end{tabular}
%\end{center}

\section{Random Generator}
\subsection{Validation}
In M/M/2/2+5 queueing system, the number of packet arriving in a fixed interval follows \textbf{Poisson Distribution} and the service time for each packet follows \textbf{Exponential Distribution}.
These input data is generated by \textit{Numpy}'s \textit{random} module. Before running the simulator, it is important to test the wellness of this random generator.

We evaluate Numpy's random generator in two ways. First we generate random values follows uniform distribution and then plot their histogram.
As shown in Figure \ref{fig:uniform}, the number of random values falling into each interval is close to each other.
This means that the generated values are very close to uniformly distributed.
Besides, we compare the normalized histogram to the `best fit' curve of both uniform distribution and normal distribution in Figure \ref{fig:uniform} and \ref{fig:normal}.
From both figures we can say that the random generator generated random values of user-specified distribution.


\begin{figure}
\centering
        \includegraphics[scale=0.6]{rg_histogram.eps}
        \caption{Histogram of Uniform Random Values Generated by Numpy}
        \label{fig:uniform}
\end{figure}
\begin{figure}
        \centering
        \includegraphics[scale=0.6]{rg_fitness.eps}
        \caption{Comparison to Best Fit Curve for Normal Distribution}
        \label{fig:normal}
\end{figure}




\section{Eliminate Warm-up Period}



\section{System Properties at Stationary State}

\subsection{Blocking Probability}


\subsection{Mean Spending Time}


\subsection{Mean Number of Packet}





% Python code embedding configuration
%\definecolor{keywords}{RGB}{255,0,90}
%\definecolor{comments}{RGB}{0,0,113}
%\definecolor{red}{RGB}{160,0,0}
%\definecolor{green}{RGB}{0,150,0}
%\lstset{language=Python,
        %basicstyle=\ttfamily\small,
        %keywordstyle=\color{keywords},
        %commentstyle=\color{comments},
        %stringstyle=\color{red},
        %showstringspaces=false,
        %identifierstyle=\color{green},
        %procnamekeys={def,class,True},
        %frame=single,
        %numbers=left,
        %numbersep=5pt,
        %numberstyle=\tiny\color{blue},
        %rulecolor=\color{black},
        %caption={Simulation Discovery Process},
        %label=lst:while-loop,
        %language=Python,
%}
%\begin{lstlisting}
%while updated:
    %for r in routers:
        %for n in r.neighbors:
            %router_n = get_router_by_name(n, routers)
            %for n_name, n_row_col in router_n.topo.items():
                %n_msg = (n_name, n_row_col)
                %r.recv_msg(n_msg)
    %updated = False
    %for r in routers:
        %if r.update_topo_database():
            %updated = True
%\end{lstlisting}

%\begin{figure}[h]
%\centering
        %\includegraphics[width=0.8\textwidth, height=0.8\textheight]{progress.ps}
%\caption{Discovery Percentage as Function of Iteration for $R_1$ and $R_6$}
%\label{fig:progress}
%\end{figure}


%%%%%%%%%%%%%%%%%%%%%%%%%%%%%%%%%%%%%%%%%
%
%     Bibliography
%
%     Use an easy-to-remember tag for each entry - eg \bibitem{How97} for an article/book by Howie in 1997
%     To cite this publication in your text, write \cite{How97}.  To include more details such as
%     page, Chapter, Theorem numbers, use the form \cite[Theorem 6.3, page 42]{How97}.
%
%\begin{thebibliography}{99}

% 
% The usual convention for mathematical bibliographies is to list alphabetically
% by first-named author (then second, third  etc. author then date)
% websites with no author names should go by the site name
%


% Typical layout for reference to a journal article
%
%\bibitem{Bovey}
%J. D. Bovey, M. M. Dodson,                         % author(s)
%The Hausdorff dimension of systems of linear forms % article name
%{\em Acta Arithmetica}                             % journal name - italics
%{\bf 45}                                           % volume number - bold
%(1986), 337--358.                                   % (year), page range

%% Typical layout for reference to a book
%%
%\bibitem{Cassels}
%J. W. S. Cassels,                                  % author(s)
%{\em An Introduction to Diophantine Approximation},% title - italics
%Cambridge University Press, Cambridge, 1965.       % Publisher, place, date.

%% Typical layout for reference to a website
%%
%\bibitem{GAP}
%The GAP Group, GAP -- Groups, Algorithms, and Programming,  % Site name
%Version 4.5.6; 2012. % other information
%(http://www.gap-system.org)  % URL


%% Typical layout for reference to an online article
%%
%\bibitem{Howie}
%J. Howie,                                            % author(s)
%{\em Generalised triangle groups of type $(3,5,2)$}, % article name - italics
%http://arxiv.org/abs/1102.2073                       % URL
%(2011).                                              % (year)
%\end{thebibliography}

\end{document}
